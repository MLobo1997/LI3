%% ================================================================================
%% This LaTeX file was created by AbiWord.                                         
%% AbiWord is a free, Open Source word processor.                                  
%% More information about AbiWord is available at http://www.abisource.com/        
%% ================================================================================

\documentclass[a4paper,portrait,12pt]{article}
\usepackage[latin1]{inputenc}
\usepackage{calc}
\usepackage{setspace}
\usepackage{fixltx2e}
\usepackage{graphicx}
\usepackage{multicol}
\usepackage[normalem]{ulem}
%% Please revise the following command, if your babel
%% package does not support en-US
\usepackage[en]{babel}
\usepackage{color}
\usepackage{hyperref}
 
\begin{document}


\begin{flushleft}
Laboratórios de Informática III
\end{flushleft}


\begin{flushleft}
Trabalho Prático no 1
\end{flushleft}


\begin{flushleft}
Relatório de desenvolvimento
\end{flushleft}





\begin{flushleft}
José Resende
\end{flushleft}





\begin{flushleft}
Miguel Lobo
\end{flushleft}





\begin{flushleft}
Patr\i{}́cia Barreira
\end{flushleft}





\begin{flushleft}
(a77486)
\end{flushleft}





\begin{flushleft}
(a78225)
\end{flushleft}





\begin{flushleft}
(a79007)
\end{flushleft}





\begin{flushleft}
1 de Maio de 2017
\end{flushleft}





\begin{flushleft}
\newpage
Conteúdo
\end{flushleft}





\begin{flushleft}
1 Introdução
\end{flushleft}





2





\begin{flushleft}
2 Concepção do problema
\end{flushleft}





3





\begin{flushleft}
3 Concepção da Solução
\end{flushleft}





4





\begin{flushleft}
4 Estruturas de Dados
\end{flushleft}





6





4.1





\begin{flushleft}
TreeHash . . . . . . . . . . . . . . . . . . . . . . . . . . . . . . . . . . . . . . . . .
\end{flushleft}





7





4.2





\begin{flushleft}
xmlArray . . . . . . . . . . . . . . . . . . . . . . . . . . . . . . . . . . . . . . . . .
\end{flushleft}





8





4.3





\begin{flushleft}
CTree . . . . . . . . . . . . . . . . . . . . . . . . . . . . . . . . . . . . . . . . . . .
\end{flushleft}





8





4.4





\begin{flushleft}
Heap . . . . . . . . . . . . . . . . . . . . . . . . . . . . . . . . . . . . . . . . . . . .
\end{flushleft}





8





\begin{flushleft}
5 Desenvolvimento das Queries
\end{flushleft}





10





\begin{flushleft}
6 Otimização
\end{flushleft}





12





6.1





\begin{flushleft}
Dificuldades e Superações . . . . . . . . . . . . . . . . . . . . . . . . . . . . . . . .
\end{flushleft}





12





\begin{flushleft}
7 Modularidade
\end{flushleft}





14





\begin{flushleft}
8 Conclusão
\end{flushleft}





15





1





\begin{flushleft}
\newpage
Cap\i{}́tulo 1
\end{flushleft}





\begin{flushleft}
Introdução
\end{flushleft}


\begin{flushleft}
Este trabalho prático tem como objetivo construir um sistema que permita analisar os artigos
\end{flushleft}


\begin{flushleft}
presentes em backups da Wikipedia, realizados em diferentes meses, e extrair informação útil desse
\end{flushleft}


\begin{flushleft}
per\i{}́odo de tempo, como por exemplo, o número de revisões ou ainda o número de novos artigos.
\end{flushleft}


\begin{flushleft}
Os problemas foram repartidos em diferentes tarefas, tarefas estas que propunham uma forma
\end{flushleft}


\begin{flushleft}
eficiente da resolução dado o volume de informação a tratar. Todo o código desenvolvido nas
\end{flushleft}


\begin{flushleft}
diferentes tarefas era submetido através do repositório GIT e avaliado na plataforma criada pelos
\end{flushleft}


\begin{flushleft}
docentes. Este relatório pretende sumarizar todos os parâmetros estudados e desenvolvidos durante
\end{flushleft}


\begin{flushleft}
este projeto, bem como os passos que foram seguidos e tomados.
\end{flushleft}





2





\begin{flushleft}
\newpage
Cap\i{}́tulo 2
\end{flushleft}





\begin{flushleft}
Concepção do problema
\end{flushleft}


\begin{flushleft}
Pela análise do problema, foi poss\i{}́vel dividir as queries em 3 grupos distintos:
\end{flushleft}





\begin{flushleft}
$\bullet$ Artigos
\end{flushleft}


\begin{flushleft}
$\bullet$ Contribuidores
\end{flushleft}


\begin{flushleft}
$\bullet$ Texto
\end{flushleft}





\begin{flushleft}
Nestes três grupos, chegou-se a conclusão que os id's é que iriam gerir a estrutura de dados.
\end{flushleft}


\begin{flushleft}
Para as queries de artigos são necessários os respetivos id's, logo torna-se imperativo desenvolver uma estrutura de dados onde a procura de um artigo seja o mais rápido poss\i{}́vel. De notar
\end{flushleft}


\begin{flushleft}
que cada artigo podia ter mais do que uma revisão, assim, é necessário implementar uma estrutura
\end{flushleft}


\begin{flushleft}
de dados associada à estrutura de cada artigo capaz de armazenar as suas revisões únicas.
\end{flushleft}


\begin{flushleft}
No caso das queries de contribuidores precisa-se de uma estrutura que guarde os nomes de
\end{flushleft}


\begin{flushleft}
contribuidores , ids e número de contribuições dos mesmos.
\end{flushleft}


\begin{flushleft}
Já para os textos, as queries focam-se no número de palavras e tamanho do texto. Por isso,
\end{flushleft}


\begin{flushleft}
será necessário guardar numa estrutura, o número de palavras , número de carateres e por fim, id
\end{flushleft}


\begin{flushleft}
e t\i{}́tulo de artigos.
\end{flushleft}





3





\begin{flushleft}
\newpage
Cap\i{}́tulo 3
\end{flushleft}





\begin{flushleft}
Concepção da Solução
\end{flushleft}


\begin{flushleft}
Decidiu-se implementar como estrutura principal uma estrutura composta por uma Hash Table,
\end{flushleft}


\begin{flushleft}
em que cada posição da Hash Table é constitu\i{}́da por uma Árvore Binária Balanceada, sendo que
\end{flushleft}


\begin{flushleft}
cada nodo desta, contém uma Lista Ligada.
\end{flushleft}


\begin{flushleft}
A escolha desta estrutura assenta nos seguinte fatores:
\end{flushleft}





\begin{flushleft}
$\bullet$ A forma como os artigos são organizados para a mesma posição da Hash Table é através de
\end{flushleft}


\begin{flushleft}
uma Árvore Binária Balanceada com fator de comparação o id. A Hash Table foi criada com
\end{flushleft}


\begin{flushleft}
intuito de reduzir o tempo de procura na Árvore Binária e, consequentemente, aumentar
\end{flushleft}


\begin{flushleft}
o número de Árvores Binárias existentes. Este aumento conduz a uma redução das suas
\end{flushleft}


\begin{flushleft}
alturas o que implica uma procura mais eficiente do artigo. Este tempo de procura em cada
\end{flushleft}


\begin{flushleft}
árvore é O(logN) (em que N é o numero de nós da árvore, ou seja, número de artigos).
\end{flushleft}


\begin{flushleft}
$\bullet$ Em cada nodo da Árvore de artigos (referida no ponto acima) será guardada informação
\end{flushleft}


\begin{flushleft}
referente às suas revisões sob a forma de lista ligada.
\end{flushleft}


\begin{flushleft}
$\bullet$ Era necessário implementar uma estrutura eficiente com travessias, com intuito de preencher
\end{flushleft}


\begin{flushleft}
estruturas auxiliares e ao mesmo tempo conseguir armazenar toda a informação necessária.
\end{flushleft}





\begin{flushleft}
Para o caso do grupo de queries referentes aos artigos apenas se utilizou a estrutura principal,
\end{flushleft}


\begin{flushleft}
visto que, através de uma simples travessia, é poss\i{}́vel obter a informação requerida.
\end{flushleft}


\begin{flushleft}
No que toca ao grupo de queries associadas aos contribuidores, foi criada uma Árvore Binária
\end{flushleft}


\begin{flushleft}
Balanceada, onde cada nodo contém uma estrutura CONT com imformação relativa aos nomes de
\end{flushleft}


\begin{flushleft}
contribuidores, ids e número de contribuições dos mesmos. Foi ainda implementada uma MaxHeap
\end{flushleft}





4





\begin{flushleft}
\newpage
com intuito de resolver a query referente aos Top contribuidores, uma vez que se trata de uma
\end{flushleft}


\begin{flushleft}
estrutura de dados bastante eficiente para este tipo de interrogação.
\end{flushleft}


\begin{flushleft}
Por fim, para o grupo das queries de texto, utilizou-se a estrutura principal e, para o caso em
\end{flushleft}


\begin{flushleft}
que as interrogações eram do tipo Top, foi implementada uma MaxHeap.
\end{flushleft}





5





\begin{flushleft}
\newpage
Cap\i{}́tulo 4
\end{flushleft}





\begin{flushleft}
Estruturas de Dados
\end{flushleft}


\begin{flushleft}
Foi utilizada a biblioteca Glib para a implementação das estruturas utilizadas neste projeto. Em
\end{flushleft}


\begin{flushleft}
baixo, encontra-se representada a estrutura que contém todas as estruturas implementadas ao
\end{flushleft}


\begin{flushleft}
longo do projeto.
\end{flushleft}





\begin{flushleft}
typedef struct TCD\_istruct \{
\end{flushleft}


\begin{flushleft}
TreeHash treeTable;
\end{flushleft}


\begin{flushleft}
xmlArray xmlInfo;
\end{flushleft}


\begin{flushleft}
CTree ContriTree;
\end{flushleft}


\begin{flushleft}
Heap TopN;
\end{flushleft}


\begin{flushleft}
Heap Top20;
\end{flushleft}


\begin{flushleft}
long topCont[10];
\end{flushleft}


\begin{flushleft}
long Atop20[20];
\end{flushleft}


\begin{flushleft}
long *AtopN;
\end{flushleft}


\begin{flushleft}
char *pref[1];
\end{flushleft}


\begin{flushleft}
long numeroArtigos;
\end{flushleft}


\begin{flushleft}
long numeroRevisoes;
\end{flushleft}


\begin{flushleft}
long numeroTotal;
\end{flushleft}


\begin{flushleft}
long numeroContributor;
\end{flushleft}


\begin{flushleft}
\}*TAD\_istruct;
\end{flushleft}





\begin{flushleft}
Nas secções seguintes será detalhada cada uma das estruturas apresentadas em cima.
\end{flushleft}





6





\newpage
4.1





\begin{flushleft}
TreeHash
\end{flushleft}





\begin{flushleft}
Para a implementação da Hash Table utilizou-se a estrutura da Glib Pointer arrays. Para Árvores
\end{flushleft}


\begin{flushleft}
Binárias Balanceadas foi utilizada a estrutura Balanced Binary Trees e, por fim, Singly-Linked
\end{flushleft}


\begin{flushleft}
Lists para listas ligadas.
\end{flushleft}


\begin{flushleft}
Posto isto, a estrutura apresenta a seguinte forma:
\end{flushleft}





\begin{flushleft}
$\bullet$ TreeHash
\end{flushleft}


\begin{flushleft}
typedef GPtrArray *TreeHash;
\end{flushleft}


\begin{flushleft}
$\bullet$ Em cada \i{}́ndice desta TreeHash
\end{flushleft}


\begin{flushleft}
typedef GTree *ArtTree;
\end{flushleft}


\begin{flushleft}
$\bullet$ Em cada nodo da árvore tem-se:
\end{flushleft}


\begin{flushleft}
typedef struct artigo \{
\end{flushleft}


\begin{flushleft}
char *id;
\end{flushleft}


\begin{flushleft}
char *titulo;
\end{flushleft}


\begin{flushleft}
RevList revisao;
\end{flushleft}


\begin{flushleft}
\}*ART;
\end{flushleft}





\begin{flushleft}
typedef GSList *RevList;
\end{flushleft}


\begin{flushleft}
$\bullet$ Em cada RevList a estrutura das revisões é implementada da seguinte forma:
\end{flushleft}


\begin{flushleft}
typedef struct revisao \{
\end{flushleft}


\begin{flushleft}
char *id;
\end{flushleft}


7





\begin{flushleft}
\newpage
char *timestamp;
\end{flushleft}


\begin{flushleft}
char *content;
\end{flushleft}


\begin{flushleft}
\}*REVISION;
\end{flushleft}





4.2





\begin{flushleft}
xmlArray
\end{flushleft}





\begin{flushleft}
Esta estrutura é usada com vista a guardar as árvores resultantes do parsing dos snapshots.
\end{flushleft}





\begin{flushleft}
typedef GPtrArray *xmlArray;
\end{flushleft}





4.3





\begin{flushleft}
CTree
\end{flushleft}





\begin{flushleft}
Estrutura que contém as informações dos contribuidores.
\end{flushleft}





\begin{flushleft}
typedef GTree *CTree;
\end{flushleft}





\begin{flushleft}
typedef struct contribuidor \{
\end{flushleft}


\begin{flushleft}
char *id;
\end{flushleft}


\begin{flushleft}
int contN; //Numero de contribuicoes
\end{flushleft}


\begin{flushleft}
char *nome;
\end{flushleft}


\begin{flushleft}
\}*CONT;
\end{flushleft}





4.4





\begin{flushleft}
Heap
\end{flushleft}





\begin{flushleft}
Para as queries cuja finalidade é a obtenção dos top's decidiu-se que a melhor implementação seria
\end{flushleft}


\begin{flushleft}
uma max-heap. Nesta, bastaria extrair a ra\i{}́z, reorganizar com vista a ficar o segundo maior na
\end{flushleft}


\begin{flushleft}
raiz e repetir este processo até se extrair o número de elementos pretendidos. A struct elemento
\end{flushleft}


\begin{flushleft}
foi criada com o intuito de reutilizar o código da heap para as três queries de top's.
\end{flushleft}





\begin{flushleft}
typedef struct elemento\{
\end{flushleft}


\begin{flushleft}
char *id;
\end{flushleft}


\begin{flushleft}
long count;
\end{flushleft}


\begin{flushleft}
\}elem;
\end{flushleft}





8





\begin{flushleft}
\newpage
typedef struct heap\{
\end{flushleft}


\begin{flushleft}
int
\end{flushleft}





\begin{flushleft}
size; /* Tamanho alocado para a Heap.*/
\end{flushleft}





\begin{flushleft}
int
\end{flushleft}





\begin{flushleft}
used; /* Número de elementos da Heap. */
\end{flushleft}





\begin{flushleft}
elem *array;
\end{flushleft}


\begin{flushleft}
\}*Heap;
\end{flushleft}





9





\begin{flushleft}
\newpage
Cap\i{}́tulo 5
\end{flushleft}





\begin{flushleft}
Desenvolvimento das Queries
\end{flushleft}


\begin{flushleft}
Para as queries all\_articles, unique\_articles e all\_revision, decidiu-se que estas deviam
\end{flushleft}


\begin{flushleft}
ser resolvidas aquando do load uma vez que à medida que os artigos e as revisões são inseridos na
\end{flushleft}


\begin{flushleft}
HashTable, facilmente se incrementava uma variável de contagem. Assim, houve a necessidade de
\end{flushleft}


\begin{flushleft}
criar 3 variáveis numeroTotal, numeroArtigos e numeroRevisões na estrutura global TADistruct
\end{flushleft}


\begin{flushleft}
que podem ser atualizadas no load e representam, respetivamente, o resultado das 3 queries. Assim
\end{flushleft}


\begin{flushleft}
o tempo de resposta a estas 3 queries passa a ser O(1).
\end{flushleft}


\begin{flushleft}
No caso das queries article\_title e article\_timestamp, basta calcular o hashcode do id
\end{flushleft}


\begin{flushleft}
recebido como argumento e, percorrer a árvore até ao nodo em questão. A segunda query requer
\end{flushleft}


\begin{flushleft}
ainda que se percorra a lista ligada até à revisão pretendida. Nestes casos ajudou bastante a criação
\end{flushleft}


\begin{flushleft}
da HashTable uma vez que a árvore tem uma altura menor.
\end{flushleft}


\begin{flushleft}
No que toca à query contributor\_name, a respetiva estrutura Árvore Binária Balanceada,
\end{flushleft}


\begin{flushleft}
será adicionada à estrutura global TAD\_istruct de modo que quando é feito o load, esta árvore
\end{flushleft}


\begin{flushleft}
seja carregada à medida que se percorre a árvore resultante do parsing. A pesquisa para encontrar
\end{flushleft}


\begin{flushleft}
o nome do contribuidor através do id será então O(logN) em que N é o numero de contribuidores(nodos).
\end{flushleft}


\begin{flushleft}
Relativamente às queries top\_10\_contributors, top\_20\_largest\_articles e
\end{flushleft}


\begin{flushleft}
top\_N\_articles\_with\_more\_words, no que refere à primeira, basta percorrer a árvore dos contribuidores guardada na estrutura TAD\_istruct e, inserir na heap os respetivos valores. Estes
\end{flushleft}


\begin{flushleft}
passos são acompanhados pela preservação do invariante em que o pai é sempre maior que os
\end{flushleft}


\begin{flushleft}
respetivos filhos. O tamanho da heap é dada pela variável numeroContributor, pertencente à estrutura TAD\_istruct. Esta variável é incrementada durante o load à medida que são inseridos os
\end{flushleft}





10





\begin{flushleft}
\newpage
contribuidores na árvore de Contribuidores. O resultado é então guardado na estrutura principal
\end{flushleft}


\begin{flushleft}
TAD\_istruct como topCont, evitando assim percorrer a árvore de contribuidores e inserir na heap,
\end{flushleft}


\begin{flushleft}
sempre que a query top\_10\_contributor é invocada. Esta metodologia adotada permite ainda
\end{flushleft}


\begin{flushleft}
que o espaço alocado para a heap possa ser libertado depois da obtenção da informação requerida.
\end{flushleft}


\begin{flushleft}
Para as restantes duas queries será aplicado o mesmo critério, contudo, para estes casos, é a Hash
\end{flushleft}


\begin{flushleft}
Table que é percorrida. Em cada revisão do artigo, é contado o tamanho e o número de palavras
\end{flushleft}


\begin{flushleft}
do texto (de uma só assentada). Estes 2 resultados são guardados num array de 2 posições e em
\end{flushleft}


\begin{flushleft}
seguida são criadas 2 max-heaps (Top20 e TopN ), uma para cada query. Evitando assim, mais
\end{flushleft}


\begin{flushleft}
uma vez, pesquisas excessivas na estrutura Hash Table.
\end{flushleft}


\begin{flushleft}
Por fim, na query titles\_with\_prefix é aplicada a função prefix\_titles\_art a cada um
\end{flushleft}


\begin{flushleft}
dos artigos, recorrendo à instrução foreach. Esta função, para ter compatibilidade com a Glib,
\end{flushleft}


\begin{flushleft}
recebe o mesmo artigo nos seus dois primeiros argumentos, o key e o value, e no último argumento,
\end{flushleft}


\begin{flushleft}
uma struct Data. Esta estrutura foi criada com o intuito de transportar:
\end{flushleft}





\begin{flushleft}
$\bullet$ [r] Apontador para lista de t\i{}́tulos que se encontra na TAD\_istruct, que será a resposta da
\end{flushleft}


\begin{flushleft}
query.
\end{flushleft}


\begin{flushleft}
$\bullet$ [used] Número de t\i{}́tulos que se encontram na lista incluindo, no final, NULL.
\end{flushleft}


\begin{flushleft}
$\bullet$ [size] Espaço dispon\i{}́vel para a inserção de t\i{}́tulos na lista (exclu\i{}́ndo o NULL).
\end{flushleft}


\begin{flushleft}
$\bullet$ [prefix] Prefixo a ser procurado.
\end{flushleft}





\begin{flushleft}
A função prefix\_titles\_art verifica se cada artigo contém o prefixo no t\i{}́tulo e, em caso
\end{flushleft}


\begin{flushleft}
positivo, adiciona-o a r, através da instrução append.
\end{flushleft}





\begin{flushleft}
Numa tentativa de otimizar a query,
\end{flushleft}





\begin{flushleft}
experimentou-se inserir os t\i{}́tulos na lista, tendo esta sido previamente ordenada recorrendo à
\end{flushleft}


\begin{flushleft}
função insertString. No entanto, não foi poss\i{}́vel obter um n\i{}́vel de eficiência superior. A melhor solução passou então por ordenar a lista no final do algoritmo através da função qsort da
\end{flushleft}


\begin{flushleft}
stdlib.
\end{flushleft}





11





\begin{flushleft}
\newpage
Cap\i{}́tulo 6
\end{flushleft}





\begin{flushleft}
Otimização
\end{flushleft}


\begin{flushleft}
Para esta secção, as escolhas mais óbvias foram a utilização da biblioteca Glib para a implementação
\end{flushleft}


\begin{flushleft}
das estruturas de dados utilizadas. E também a utilização da API OpenMP para a introdução
\end{flushleft}


\begin{flushleft}
de paralelismo. Outro aspeto considerado importante foi a utilização de uma Max-Heap como
\end{flushleft}


\begin{flushleft}
estrutura auxiliar das queries que envolviam o prefixo top's.
\end{flushleft}





6.1





\begin{flushleft}
Dificuldades e Superações
\end{flushleft}





\begin{flushleft}
A utilização da Glib levantou algumas dificuldades. De realçar a composição entre as estruturas,
\end{flushleft}


\begin{flushleft}
nomeadamente entre Hash Table onde em cada posição desta temos uma Árvore Binária Balanceada e em cada nodo existe ainda uma Lista Ligada. Isto aliado ao facto do grupo de trabalho não
\end{flushleft}


\begin{flushleft}
estar habituado a implementar e utilizar funções genéricas, introduziu uma certa complexidade à
\end{flushleft}


\begin{flushleft}
resolução do problema.
\end{flushleft}


\begin{flushleft}
Já em relação ao OpenMP a principal dificuldade residiu em encontrar partes do código que
\end{flushleft}


\begin{flushleft}
não sejam interdependentes. Assim, decidimos unicamente aplicar a diretiva OpenMP às árvores
\end{flushleft}


\begin{flushleft}
criadas no parsing dos inúmeros ficheiros (snapshots) através da xmllib, onde é criada uma árvore
\end{flushleft}


\begin{flushleft}
para cada snapshot. De notar que tudo isto só foi poss\i{}́vel graças à compatibilidade da xmllib
\end{flushleft}


\begin{flushleft}
com paralelismo.
\end{flushleft}





\begin{flushleft}
xmlArray parseMult(xmlArray array, int N, char *filename[N])\{
\end{flushleft}


\begin{flushleft}
int i;
\end{flushleft}


\begin{flushleft}
if(array != NULL)\{
\end{flushleft}


\begin{flushleft}
\#pragma omp parallel for
\end{flushleft}


12





\begin{flushleft}
\newpage
for(i = 0 ; i $<$ N ; i++)
\end{flushleft}


\begin{flushleft}
g\_ptr\_array\_add(array, parse(filename[i]));
\end{flushleft}


\begin{flushleft}
g\_ptr\_array\_set\_size (array, N);
\end{flushleft}


\}


\begin{flushleft}
return array;
\end{flushleft}


\}





\begin{flushleft}
Assim consegui-se um real time de aproximadamente 6segundos e um user time de aproximadamente 10segundos. Houve uma melhoria do real time e um aumento no user time porque o
\end{flushleft}


\begin{flushleft}
processo está ligado ao CPU e aproveita a execução paralela em vários núcleos/CPUs, enquanto
\end{flushleft}


\begin{flushleft}
que sem openMP o real time era aproximadamente igual ao user time porque o processo está ligado
\end{flushleft}


\begin{flushleft}
ao CPU e não tem qualquer vantagem de execução em paralelo.
\end{flushleft}


\begin{flushleft}
Por último em relação à Max-Heap, esta foi escolhida com intuito de evitar criar um Array
\end{flushleft}


\begin{flushleft}
com todos os id's e o número que dita a ordem (e.g. número de contribuições dos contribuidores)
\end{flushleft}


\begin{flushleft}
e consequentemente ordena-lo para retornar os n maiores. A Max-Heap é então uma melhor
\end{flushleft}


\begin{flushleft}
alternativa.
\end{flushleft}





13





\begin{flushleft}
\newpage
Cap\i{}́tulo 7
\end{flushleft}





\begin{flushleft}
Modularidade
\end{flushleft}


\begin{flushleft}
Atendendo à dimensão do projeto, foi necessária uma organização cuidada do código para que
\end{flushleft}


\begin{flushleft}
este seja controlável e leg\i{}́vel, em todas as suas fases (desenvolvimento, teste e manutenção). O
\end{flushleft}


\begin{flushleft}
projeto foi dividido em 3 partes essenciais: load, init e parse. No load os dados são carregados
\end{flushleft}


\begin{flushleft}
para a estrutura, no init as estruturas são inicializadas e, por fim, no parse é feito o parsing dos
\end{flushleft}


\begin{flushleft}
snapshots através da biblioteca xmlib.
\end{flushleft}


\begin{flushleft}
As queries foram organizadas nos seus 3 grandes tipos: artigos, contribuidores e textos. As
\end{flushleft}


\begin{flushleft}
estruturas auxiliares surgem num novo módulo bem como o clean.
\end{flushleft}





14





\begin{flushleft}
\newpage
Cap\i{}́tulo 8
\end{flushleft}





\begin{flushleft}
Conclusão
\end{flushleft}


\begin{flushleft}
As queries implementadas respondem aos requisitos impostos pelo enunciado, pelo que este grupo
\end{flushleft}


\begin{flushleft}
de trabalho se considera sobremaneira satisfeito. É no entanto, evidente, que muitas outras otimizações poderiam ser acrescentadas, tanto na função que conta as palavras e tamanho de um
\end{flushleft}


\begin{flushleft}
texto como uma maior aposta no paralelismo. Seria portanto desejável, como exerc\i{}́cio futuro,
\end{flushleft}


\begin{flushleft}
implementar a função count, integrando, por exemplo, a diretiva openMP, etc.Existem portanto
\end{flushleft}


\begin{flushleft}
diversas formas de continuar o trabalho que aqui foi produzido, todas elas revestidas de utilidade
\end{flushleft}


\begin{flushleft}
prática.
\end{flushleft}





15





\newpage



\end{document}
